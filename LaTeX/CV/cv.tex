% !Mode:: "TeX:UTF-8"
%----------------------------------------------------------------------------------------
% Definições gerais e packages
%----------------------------------------------------------------------------------------
% Doc definition
\documentclass[12pt, A4]{article}
% Font basics
\usepackage[T1]{fontenc}
% Font calibri
\usepackage{fontspec}
\setmainfont{calibri}
\usepackage{microtype} % Justificação melhorada
% Imagens
\usepackage{graphicx}
% Listas
\usepackage{listings}
\usepackage{enumitem}
% Unir os pdf's
\usepackage{pdfpages}
% Língua portuguesa
\usepackage{babel}
% Hipenation rules
\usepackage{hyphenat}
% Page layout
\usepackage{paracol} % Colunas justificadas
\usepackage[a4paper]{geometry}
\geometry{top=0.75cm, bottom=0.75cm, left=0.75cm, right=0.75cm} % Margens
\thispagestyle{empty} % Para não numerar a página
\setlength{\parindent}{0mm} % Sem indentação
\frenchspacing % Espaçamento após frases normal
% Repeat
\usepackage{expl3}
\ExplSyntaxOn
\cs_new_eq:NN \Repeat \prg_replicate:nn
\ExplSyntaxOff
% Colours
\usepackage{afterpage} % Tirar cor nas páginas seguintes
\usepackage{transparent}
\usepackage{color}
\usepackage{xcolor}
\definecolor{titlecol}{RGB}{0, 100, 160} % azul
\definecolor{iconcol}{RGB}{0, 100, 160} % azul
\definecolor{circcol}{RGB}{140, 140, 140} % cinza
\definecolor{localcol}{RGB}{75, 75, 75} % cinza
\definecolor{pagebackcol}{RGB}{205, 205, 205} % cinza
% Icones
\usepackage{fontawesome5}
\usepackage{amssymb}
% Hyperlinks (deve ficar no final de todas as packages)
\usepackage{hyperref}
\usepackage{algpseudocode}
\hypersetup{
colorlinks=true,
linkcolor=blue,
filecolor=magenta,
urlcolor=blue,
pdftitle={Nome do PDF},
pdfpagemode=FullScreen,
}
\urlstyle{same}
%----------------------------------------------------------------------------------------
% Comandos CV
%----------------------------------------------------------------------------------------
% Texto "justificado" nas colunas
\newcommand{\mpwidth}{\linewidth-\fboxsep-\fboxsep}
\newcommand{\cvtext}[1]{
    \begin{tabular*}{1\mpwidth}{p{1\mpwidth}}
		\parbox{1\mpwidth}{\large{#1}}
	\end{tabular*}
}
% Formatação dos títulos
\newcommand{\cvtitle}[1]{
    \cvtext{
        \textbf{{\Large{\textcolor{titlecol}{\uppercase{#1}}}}}
    }
}
% Formatação dos títulos da página 2
\newcommand{\cvtitlex}[1]{
        \textbf{{\Large{\textcolor{titlecol}{\uppercase{#1}}}}}
}
% Formatação do nome da formação\função
\newcommand{\cvdegree}[1]{
    \cvtext{
        {\textbf{\uppercase{#1}}}
    }
}
% Formatação dos nomes dos locais
\newcommand{\cvlocal}[1]{
    \cvtext{
        {\textcolor{localcol}{\uppercase{#1}}}
    }
}
% Separadores
\newcommand{\cvsep}{
    {\textcolor{titlecol}{ \rule{0.999\columnwidth}{1.25pt} }}
}
% Icones
\newcommand{\icone}[1]{
    {\color{iconcol}{\faIcon{#1}}}
}
% Medida das habilidades
\newcommand{\skill}[1]{
    \small{
        \Repeat{#1}{\color{titlecol}{\faIcon{circle}} }\Repeat{5-#1}{\color{circcol}{\faIcon{circle}} }
    }
}
% Distância entre elementos principais esquerda
\newcommand{\elsepe}{
    {\\[25pt]}
}
% Distância entre elementos principais direita
\newcommand{\elsepd}{
    {\\[40pt]}
}
% Distância entre títulos e texto
\newcommand{\txsep}{
    {\\[10pt]}
}
% Distância entre formações
\newcommand{\degsep}{
    {\\[8pt]}
}
% Distância entre formações e respetivos locais e datas
\newcommand{\locsep}{
    {\\[4pt]}
}
% Distância entre skills
\newcommand{\skisep}{
    {\\[0.75pt]}
}
%----------------------------------------------------------------------------------------
% Conteúdo CV
%----------------------------------------------------------------------------------------
\begin{document}
% Página 1
\columnratio{0.36} % Rácio de colunas
\setlength{\columnsep}{9pt} % Distância separação entre as colunas
\begin{paracol}{2}
% Coluna da esquerda
\begin{leftcolumn}
% Foto
\includegraphics[width=\columnwidth]{path/to/portrait.png}\\
% Separador
\cvsep\\[7pt]
% Nome
\cvtitle{Apelido,\\[4pt]Nome}\\[7pt]
% Separador
\cvsep\\[20pt]
% Contactos
\cvtitle{Contactos}
\txsep
\cvtext{
    \icone{envelope}\ \ mail@domain.com\\[2pt]
    \icone{mobile-alt}\ \ \ phone number\\[2pt]
    \icone{linkedin}\ \ \href{https://www.linkedin.com/in/...}{/in/username}\\[2pt]
    \icone{github}\ \ \href{https://github.com/...}{@username}}
\elsepe
% Competências gerais
\cvtitle{Competências Gerais}
\txsep
\cvtext{
    Competência I • Competência II • Competência III • 
    Competência IV • Competência V • Competência VI • 
    Competência VII • Competência VIII • Competência IX •}
\elsepe
% Competências digitais
\cvtitle{Competências Digitais}
\txsep
\cvtext{
    \textbf{C++\hfill{\skill{4}}\skisep Python\hfill{\skill{5}}\skisep Linux\hfill{\skill{5}}\skisep Go (Golang)\hfill{\skill{3}}\skisep SQL\hfill{\skill{4}}\skisep LaTeX\hfill{\skill{1}}\skisep MS Office\hfill{\skill{2}}\skisep VBA\hfill{\skill{2}}\skisep}}
\end{leftcolumn}
% Coluna da direita
\begin{rightcolumn}
% Experiência
\setlist[itemize]{align=parleft,left=0pt..1em, itemsep=0.25pt, topsep=2pt,
    label=\scriptsize{\footnotesize{\faIcon{caret-right}}}, labelsep=-15pt, leftmargin=15pt} % Definições lista
\cvtitle{Experiência}
\txsep
    \cvdegree{Área etc}
    \locsep
        \cvlocal{Firma, S.A.}
        \locsep
            \cvtext{
                Fevereiro de 2022 – Julho de 2025
                \begin{itemize}
                    \item Analisou diversos dados
                    \item Inspecionou várias coisas
                    \item Elaborou tal
                    \item Apoiou tais processos
                    \item Realizou xyz
                \end{itemize}
            }
\elsepd
% Educação
\cvtitle{Educação}
\txsep
    \cvdegree{Mestrado em x}
    \locsep
        \cvlocal{Faculdade y}
        \locsep
            \cvtext{Janeiro de 1999 - Março de 2001}
    \degsep
    \cvdegree{Licenciatura z}
    \locsep
        \cvlocal{Faculdade j}
        \locsep
            \cvtext{Julho de 1997 - Novembro de 1998}
\elsepd
% Educação complementar
\cvtitle{Educação Complementar}
\txsep
    \cvdegree{Curso tal}
    \locsep
        \cvlocal{Faculdade i}
        \degsep
    \cvdegree{Certificado x}
    \locsep
        \cvlocal{Instituto y}
        \degsep
    \cvdegree{Curso z}
    \locsep
        \cvlocal{Instituto j}
\elsepd
\end{rightcolumn}
\end{paracol}
% Página 2
\newpage
\thispagestyle{empty}
\newgeometry{margin=1in} % Alterar as margens da próxima página
\setlist[itemize]{align=parleft,left=0pt..1em, itemsep=3pt, topsep=8pt,
    label=\scriptsize{\scriptsize{\faIcon{link}}}, labelsep=-12pt, leftmargin=11.5pt}
% Anexos
\cvtitlex{Anexos}
    \begin{itemize}
        \item \hyperlink{page.3}{\underline{Diploma do Mestrado}}
        \item \hyperlink{page.4}{\underline{Diploma da Licenciatura}}
        \item \hyperlink{page.5}{\underline{Diploma do Curso x}}
        \item \hyperlink{page.6}{\underline{Carta de recomendação de y}}
        \item \hyperlink{page.7}{\underline{Certificados da conferência z}}
        \elsepe
    \end{itemize}
% Links úteis
\cvtitlex{Links úteis}
    \begin{itemize}
        \item \href{https://domain.com}
        {\underline{domain.org(comprovativo cetificado x)}}
        \item \href{https://domain2.com}
        {\underline{domain2.org(comprovativo diploma y)}}
        \item \href{https://verify.domain3.com/}
        {\underline{Verificação de diploma z}}
    \end{itemize}
% Páginas restantes
\afterpage{\nopagecolor} % Tirar cor de fundo das páginas seguintes
\includepdf[pages=-]{path/to/Appendix.pdf}
\end{document}
